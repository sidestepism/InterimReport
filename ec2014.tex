%%
%% 研究報告用スイッチ(情報処理学会用ファイルをEC2014用に変更)
%% [techrep]
%%
%% 欧文表記無しのスイッチ(etitle,jkeyword,eabstract,ekeywordは任意)
%% [noauthor]
%%
\documentclass[submit,techreq]{ec2014}
%\documentclass[submit,techreq,noauthor]{ipsj}

\usepackage[dvips]{graphicx}
\usepackage{latexsym}

\def\Underline{\setbox0\hbox\bgroup\let\\\endUnderline}
\def\endUnderline{\vphantom{y}\egroup\smash{\underline{\box0}}\\}
\def\|{\verb|}

\setcounter{巻数}{55}%vol53=2012
\setcounter{号数}{10}
\setcounter{page}{1}

\def\newblock{}
\def\url{}


\begin{document}


\title{笑い声呈示により自然な笑顔を撮影するカメラの提案}
\etitle{A camera system for taking a picture of natural smile \\ with presenting laughter sound}

\affiliate{UTokyo}{東京大学\\
UTokyo, 7-3-1, Hongo, Bunkyo-ku, Tokyo}

\author{伏見 遼平}{Ryohei Fushimi}{UTokyo}[fushimi@nae-lab.org]
\author{福嶋 政期}{Shogo Fukushima}{UTokyo}[shogo@nae-lab.org]
\author{苗村 健}{Takeshi Naemura}{UTokyo}[naemura@nae-lab.org]

\begin{abstract}
記念撮影で自然な表情を撮影するのは難しい.本研究では被撮影者に自然な笑顔の表出を促すカメラを検討している.本稿では笑い声が笑顔や笑いを誘発する現象に着目し,シャッターを切る前に笑い声を再生することで自然な笑顔を撮影するカメラシステムを提案する.効果を検証するために,シャッター音・笑い声(青年)・笑い声(幼児)の3条件で表情の変化を比較した.さらに笑い声呈示から表情表出までの時間差に着目しシャッターを切るタイミングを検討した.
\end{abstract}

%\begin{jkeyword}
%情報処理学会論文誌ジャーナル,\LaTeX,スタイルファイル,べからず集
%\end{jkeyword}

\begin{eabstract}
It is hard to take a "natural" smile. Even if you ask a model to make smile or say "cheese!", what you can take is voluntary, unnatural smile. From the fact that laughter voice can induce smiles and laughter, we proposed a camera system which can cause a natural smile (Duchenne smile) using laughter. We compared the difference of effects with three sounds: laughter of a man, laughter of a baby and the sound of a shutter.
\end{eabstract}

%\begin{ekeyword}
%IPSJ Journal, \LaTeX, style files, ``Dos and Dont's'' list
%\end{ekeyword}

\maketitle

%1
\section{はじめに}

記念撮影やポートレートなどの人物写真において,自然な表情を撮影するのは難しい.「ハイチーズ」「笑って!」など,カメラマンが被撮影者に言語的な働きかけを行うことも多いが,これにより笑顔を作るように誘導したとしても,作り笑いになることが多い.

% (「自然な笑顔」ぼくたちが自然な笑顔をどういうものだと思っているかを示す.ハイチーズみたいな笑い方,自然な表情であるかは微妙.イラストでこれやりたいが分かる2枚)
% カメラは被写体のありのままの姿を記録するためにデザインされている.ストロボなどを除けば、基本的にはカメラや付属する機器は被写体に対して働きかけを行うものは少ない.しかし、ポートレートなどの人物写真については、被写体となる人間とのコミュニケーションも重要である.カメラマンからは「ハイチーズ」「笑って」などの声掛けがなされるが,言語でコミュニケーションを取ったり,笑顔を作るように誘導したとしても,随意的な笑顔しか撮影することができない.

\begin{figure}[b!]
\begin{minipage}{0.49\columnwidth}
\begin{center}
\includegraphics[width=35mm, bb=0 0 572 834]{images/cap_28.jpg}
\end{center}
\end{minipage}
\begin{minipage}{0.49\columnwidth}
\begin{center}
\includegraphics[width=35mm, bb=0 0 572 834]{images/cap_123.jpg}
\end{center}
\end{minipage}
\label{hirashimasmile}
\begin{center}
\caption{左: 作り笑いの笑顔 右: 本研究の目指す自然な笑顔}

\end{center}
\end{figure}


本研究では、カメラそのものが被写体の情動に直接働きかけ、笑顔を誘発するようなカメラシステムを研究している.笑顔を誘発するために,つられ笑いや思い出し笑いなどの非随意的なメカニズsムを使えば,自然な笑顔が撮影できる可能性がある.図\ref{hirashimasmile}に本研究のイメージする参考画像を示した.

中でも本稿では、他者の笑いが笑いを誘発する効果を利用して、シャッターを切る前に笑い声を再生することで自然な笑顔を撮影するカメラシステム"爆笑カメラ"を提案する.また,システムをスマートフォン上で動作するアプリケーションとして実装し,実際に効果を検討した.

まず,予備実験として事前に収集した笑顔を誘発しやすい笑い声音声素材の中から最も笑顔の誘発に効果的である素材を選定した.次に実験室の統制された環境で,2種類の笑い声音声と通常のシャッター音で撮影された笑顔にどのような違いがあるかを調べた.最後に,スマートフォンアプリとして配布し,実際の記念撮影のシチュエーションでどのような写真が撮影できたかを調べた.

%2
\section{関連研究}

ここでは,人の笑顔のメカニズムに関する知見を整理し,我々が目指す笑顔の種類について言及する.次に写真撮影のシーンにおけて,被撮影者に働きかける事例を撮影者から被撮影者に働きかける事例とカメラシステムから被撮影者に働きかける事例に分類する.さらにそれぞれの分類の中で,働きかけの方法を言語的なものと非言語的なものに分け,それぞれの特長を整理した.

\subsection{人の笑顔のメカニズムに関する知見}

Duchenneは,人の笑顔には2つの異なるタイプがあると指摘した.大頬骨筋と眼輪筋両方の動きが観察されるDuchenne Smileと,大頬骨筋のみの動きしか見られないnon-Duchenne Smileである\cite{de1990mechanism}.このうち眼輪筋は不随意であり,Duchenne Smile こそが真の笑い,「魂の喜ばしい衝動」であり,non-Duchenne Smileは愛想笑いの表出であるとしている.Surakkaらは前者と後者の筋肉の動きの差をEMGを用いて調査し,それぞれに対応する筋肉群をまとめた \cite{surakka1998facial}.これら2種の笑顔の表出は,単に使われる筋肉が異なるというわけだけではなく,表情筋の情動性の制御を担う脳の経路は,同じ筋肉の随意的な制御を担う部位とは異なることもわかっている\cite{tanaka201007}.

我々が撮影することを目指すのはDuchenne Smile,すなわち意識して表出できるものではない不随意の眼輪筋の収縮を伴う笑顔である.

\subsection{撮影者から被撮影者への働きかけ}

記念撮影のシーンにおいてカメラマンによる被写体への語りかけや掛け声が多くみられ,これには位置やポーズ,表情についての指示だけではなく,被撮影者をリラックスさせようという意図の語りかけも多く見られる.このような言語的アプローチは非撮影者に表情の指示を細かく伝えることができる点で有効だが,これは随意的な笑顔しか撮影できない.トークなどを通じて非随意的な笑いを意識して引き出すことのできるカメラマンも存在するが,個々のスキルに依存する.

「チーズ」「ミッキー」などの語末に /i/ 音がある単語を発音させ,口角を上げさせる撮影の手法もある.これは随意的な笑いではあるものの,笑顔を撮影するのに有効な方法である,

\subsection{カメラシステムから被撮影者への働きかけ}


市販カメラのうち Canon Kiss X の一部機種では,シャッター音を設定できるものがある\cite{canonkissx}.ガラケーと呼ばれる日本製フィーチャーフォンでもシャッター音を設定できる機能がある.これらは実際のシャッターが下りるのと同時に音声を鳴らすものであり,音声再生によって撮りたい表情を引き出すという本研究の狙いとは異なるものである.

プリクラ撮影機では,具体的な表情やポーズ等の指示について事前収録された音声が再生される.この指示は意識的な笑顔や"変顔"を表出させる言語的な働きかけである.プリクラ撮影機では他にも音楽や照明など撮影シーンを盛り上げる様々な試みがなされているが,「思わず笑顔させる」ことを意図したものではない.

D2Cのリリースしたスマートフォン向けアプリ「笑顔が撮れる こどもカメラ」は,幼児の自然な笑顔を写真におさめるためのアプリケーションである.スマートフォンの画面にキャラクターを表示させ,これを動かすことで幼児の興味を引き,シャッターを切る.これは撮影しにくい子供の笑顔を気を引くことにより撮影するためのものであるが,本研究には最も近い事例である.ただし,笑顔を誘発するための検討はされていない.

%\begin{figure}[h!]
%  \centering  
%\includegraphics[width=55mm, bb=0 0 680 455]{images/murataphoto-main.jpg}
%\caption{プロカメラマンの写真撮影の例}
%  \label{pro-cameraman}
%\end{figure}

%% ※ 表

\subsection{笑い声呈示による笑い誘発}

我々は,カメラシステムからの働きかけの中で非随意な笑顔を誘発するために,他者ん笑い声を利用する事を検討する.笑い声の伝染現象は古くから研究されている.Provineは,この伝染現象は学習ではなく無意識的で生得的な行動であることを指摘している\cite{provine1996laughter}.またProvineは,心理学の講義を履修した128名の学生に対し,18秒間の笑い声を聞かせ,42秒間の間を空けることを10回繰り返し,それぞれの試行について笑顔になったか否かおよび笑ったか否かを報告させた\cite{provine1992contagious}.繰り返しにより笑顔・笑いを誘発できた割合は大きく下がり,繰り返しの後半では不快感を感じたという報告も多かった. Platowらは,社会的に近い集団による笑い声の方が,そうでない集団の笑い声よりも笑いの伝染を引き起こしやすいことを指摘している\cite{platow41n}.

笑い声を音声や映像コンテンツへ応用した事例は多く存在し,一般的にラフトラックと呼ばれる.

最後に,笑い誘発を実世界のデバイスに応用した例について取り上げる.「笑い袋」は,1969年ごろ流行したおもちゃで,ボタンを押すとシュールな笑い声がとめどなく流れるというものである.「くすぐりエルモ (Tickle me Elmo)」\cite{ticklemeelmo}は,子供に人気のキャラクター「エルモ」のぬいぐるみであるが,腹部を触るとエルモが笑い転げる.我々はこれらからインスピレーションを受け,「笑い増幅器」として実装した\cite{fukushima2010}.嶋本らは,プレゼンテーション時に聴衆のPCから笑い声を再生し,笑いや拍手を誘発するシステムを提案している.我々はこの誘発メカニズムをこれをカメラ撮影のインターフェイスに応用したシステムを提案する.

\section{システム}

\begin{figure}[h]
\begin{minipage}{0.49\columnwidth}
\begin{center}
\includegraphics[width=35mm, bb=0 0 434 704]{images/ss/ss1.png}
\caption{スクリーンショット}
\label{ss1}
\end{center}
\end{minipage}
\begin{minipage}{0.49\columnwidth}
\begin{center}
\includegraphics[width=35mm, bb=0 0 434 704]{images/ss/ss1.png}
\caption{スクリーンショット}
\label{ss2}
\end{center}
\end{minipage}
\end{figure}

本システムでは,笑い声を非撮影者に呈示しながら写真を撮影する必要がある.一眼レフなどのデジタルカメラのシャッターと同期して音声を再生する装置を検討したが,同期のための機構を特別に必要となってしまう.また,複雑な装置のセットアップが必要なシステムでは,結局のところ「誰でも自然な笑顔が撮影できる」という目的に適わない.同期を簡便に実現でき,音声再生から撮影までの遅延を自由に制御でき,さらに簡単にセットアップできるという利点を持つスマートフォン向けのアプリケーションとして実装した.図\ref{ss1}, \ref{ss2} にスクリーンショットを示した.

本システムはスマートフォン,スピーカによって構成されている.ボタンを押した時刻から録画を始め,タイマーで,音声を再生し,録画をストップするアプリケーションをObjective-Cで記述し,iOSのスマートフォンにインストールした.アプリケーションは通常の写真撮影モードと,動画撮影するモードがある.また,事前にインストールした音声を選択し,再生しながら写真や動画の再生を行うことができる.

将来的に,一般向けの写真撮影アプリケーションとして配布し,データを収集することを見込み,著者以外でも実験ができるように教示をアプリケーションの画面の中に埋め込んだ.

\section{予備実験}

本実験で使用する音声の種類を選定するために予備実験を行った.下記に示す5種類の笑い声を用意し,3人の被験者 (男性2名,女性1名) に聴かせて,表情を観察し感想を聞いた.笑顔を誘発する効果の大きかった (1)幼児の笑い声 (3)青年の笑い声を本実験に用いることにした.音声データはロイヤリティフリーの音声素材を提供するWebサイトaudioblocks\cite{AudioBlocks}から取得した.

\begin{enumerate}
 \item 幼児の笑い声 (12秒)
 \item 少年の笑い声 (10秒)
 \item 青年の笑い声 (15秒)
 \item 青年の3名の笑い声 (11秒)
 \item 多人数の笑い声 (ラフトラック) (12秒)
\end{enumerate}

\section{実験}

\subsection{目的}

システムが有効に笑顔を誘発できているかを確かめるため,実験室で本システムを使って撮影した画像群について,コンピュータビジョンを用いた顔認識API「Rekognition API」を用いて笑顔尺度を測定し,システムの効果を検証するとともに,もっとも効果が現れるタイミング,音声の種類と男女要因の交互作用を調査した.

\subsection{実験条件および実験設備環境}

\begin{figure}[h!]
  \centering  
\includegraphics[width=65mm, bb=0 0 2404 2404]{images/system.jpg}
\caption{実験用にセットアップしたシステム}
  \label{recursive}
\end{figure}

21-36歳の被験者19名に対して実験を行った.実験は静かな実験室で,実験者と二人の状況で行われた.実験条件として,音声を予備実験で選定した「青年の笑い声」「幼児の笑い声」「一眼レフカメラのオートフォーカス合焦音+シャッター音(対象条件)」に設定した3条件で実験を行った.カメラデバイスをスタンドに設置し,実験者はスタンドの右斜め後ろで被験者に教示を与えたのち,デバイスを操作した.

被験者の属性は東京大学・東京外国語大学の学生および東京大学の大学職員であった (男性8名, 女性11名, 平均23.1歳) .

\begin{figure}[h!]
  \centering  
\includegraphics[width=60mm, bb=0 0 2312 2312]{images/DSC05173.jpg}
\caption{実験の様子}
  \label{recursive}
\end{figure}

\begin{table}[htb]
  \begin{center}
    \caption{実験条件}
    \begin{tabular}{|l|c|r||r|} \hline
      条件1 & 青年の笑い声 \\ \hline 
      条件2 & 幼児の笑い声 \\ \hline
      条件3 & 一眼レフカメラのAF合焦音+シャッター音 (対照) \\ \hline
    \end{tabular}
    \label{tab:price}
  \end{center}
\end{table}

\subsection{手続き}

被験者には,この実験は写真を撮る際のシャッター音の表情への影響を調べる実験であることが伝えられた.さらに注意としてシャッター音には長いものも短いものもあること,撮影中はなるべくカメラレンズを見つめるようにすることを伝えたのち,3条件で撮影を行った.順序効果を相殺するため,実験条件の順序はラテン方格法により割り当てられた.3条件すべての撮影後,それぞれの体験についてどういう感想を持ったかを聞いた.

\subsection{評価方法}

撮影された8秒間の動画から,15fpsで静止画像を切り出し,条件間の表情の差異,および表情の時間変化について分析を行った.

まず,Rekognition APIを用いて,各画像について笑顔尺度の測定を行った.Rekognition APIはOrbeus社による表情認識APIで,画像を送信すると映っている表情について笑顔尺度を解析し,0〜100の値で結果を返す.この値を顔全体の笑顔の度合いとみなし,値の時間変化と条件間の差異を分析した.

ただし,目を閉じている場合,笑顔尺度は目を閉じていない前後のフレームよりも大きく値が低下する.このことから,目を閉じていると判定されたフレームと,その前後2フレームについては,3フレーム前の笑顔尺度の値と,3フレーム後の笑顔尺度の値の平均値を採用した.

\subsection{結果と考察}

笑顔尺度の時間変化を条件ごとに平均した値をプロットしたものを図\ref{graph-smooth}に示す.音声が流れ始めてすぐは全く変化がなく,1.0秒ごろから笑顔度に条件間の差が認められる.

\begin{figure}[h!]
  \centering  
\includegraphics[width=90mm, bb=0 0 600 450]{images/smooth5_avg.png}
\caption{笑顔尺度の値の変化}
  \label{graph-smooth}
\end{figure}


\begin{figure}[h!]
  \centering  
\includegraphics[width=90mm, bb=0 0 600 450]{images/graph-diff.png}
\caption{0-1sと1-2sの笑顔尺度の差の平均値}
  \label{graph-avg}
\end{figure}

実際に効果が出ていることを確かめるため,0.0秒〜0.6秒の笑顔度の値を作り笑いの笑顔の値,1.0〜2.0秒の笑顔度の値の平均値を音声によって誘発された笑顔の値の代表値とした.誘発笑いの笑顔度と作り笑いの笑顔度の差を Shapiro-Wilk の方法を用いて正規性を検定したところ,正規性は認められなかった(p < .05).条件内で作り笑い-誘発笑いの間に差があるかどうかウィルコクソンの符号順位検定を用いて検定した結果,作り笑い-誘発笑いの笑顔度の差は条件1,2で有意 ($p<.02$) だが,条件3では有意ではなかった.

さらに (誘発笑いの笑顔度) - (作り笑いの笑顔度) の差について,各条件群について対応あり3群の分散分析を行った結果,$p<0.05$で有意に差があるという結果が出た.2条件間で検定を行ったところ,条件1-条件3,条件2-条件3について有意に差があり,Bonferroniの補正を考慮しても条件2-条件3間では$p<0.05$で優位に差があった.

\begin{table}[htb]
  \begin{center}
    \caption{対応あり分散分析の結果 (p値, カッコ内は Bonferroni の補正)}
    \begin{tabular}{|l|c|r||r|} \hline
      3群 & $0.025$  \\ \hline 
      条件1-2 & $0.760 (2.280)$ \\ \hline
      条件2-3 & $0.004 (0.012)$ \\ \hline
      条件1-3 & $0.041 (0.123)$ \\ \hline
    \end{tabular}
    \label{tab:price}
  \end{center}
\end{table}

また,笑い誘発の男女差を確認するため,各条件の (誘発笑いの笑顔度) - (作り笑いの笑顔度) の値についての男女差を,マン・ホイットニーのU検定を用いて検定した結果を表\ref{manw}に示す.条件1 (青年の笑い声) については女性のほうが,条件3 (シャッター) については男性のほうが,それぞれ有意に笑顔が誘発されやすい(いずれも$p < .05$)という結果となった.

\begin{table}[htb]
  \begin{center}
    \caption{マン・ホイットニーのU検定の結果 (p値)}
    \begin{tabular}{|c|c|c|}  \hline
      条件1 (青年) & $0.037$ & (男性) $<$ (女性) \\ \hline
      条件2 (幼児)  & $0.156$ & (男性) $<$ (女性) \\ \hline
      条件3 (シャッター) & $0.046$ & (男性) $>$ (女性) \\ \hline
    \end{tabular}
    \label{manw}
  \end{center}
\end{table}


さらに,音声の種類の違いが笑顔誘発度合いに及ぼす影響の男女差を調べるため,各被験者について2条件間の誘発笑いの笑顔度の差について,男女差があるかを,マン・ホイットニーのU検定を用いて確認した.条件1(青年) の誘発笑いの笑顔度と,条件2(幼児)の誘発笑いの笑顔度の符号を考慮した差は,女性のほうが有意に大きい ($p < .05$)が
ほかの2条件間では男女差は見られなかった.この結果は女性は比較的青年よりも幼児の笑い声に笑いを誘発され,男性はその逆の傾向があることを示唆する.

%結果を\ref{pair-genderdiff}に示す.

%\begin{table}[htb]
%  \begin{center}
%    \caption{マン・ホイットニーのU検定の結果 (p値)}
%    \begin{tabular}{|l|c|r||r|} \hline
%      (条件1) - (条件2) & $0.024$ & (男性) $>$ (女性) \\ \hline
%      (条件2) - (条件3) & $0.266$ & (男性) $<$ (女性) \\ \hline
%      (条件3) - (条件1) & $0.056$ & (男性) $>$ (女性) \\ \hline
%    \end{tabular}
%    \label{pair-genderdiff}
%  \end{center}
%\end{table}

最後に,代表的な画像を下記に示す.いずれも音声が流れ始めてから3.0秒後の写真である.

\begin{figure}[h!]
\begin{minipage}{0.49\columnwidth}
\begin{center}
\includegraphics[width=35mm, bb=0 0 663 834]{images/nagatatsucap_15.jpg}
\caption{条件2}
\end{center}
\end{minipage}
\begin{minipage}{0.49\columnwidth}
\begin{center}
\includegraphics[width=35mm, bb=0 0 663 834]{images/nagatatsucap_41.jpg}
\caption{条件3}
\end{center}
\end{minipage}
\end{figure}


\subsection{内観報告}

女性2名から,男性の笑い声は不快だという意見があった.また女性2名から,男性青年の笑い声よりも赤ちゃんのほうが笑いやすいという意見があった.そのうち1名は,それは自分が女性だからではないかという意見を付した.これらの報告は,実際に女性では青年より幼児の方が笑顔度の誘発度合いが大きかったことと符合する.

また男性のうち2名は,「いつも写真に撮られるくらいの笑顔で映ってください」という教示に対し,「いつも写真には笑顔では映らない」と応えたため,「それでは,いつも写真に取られる表情で写ってください」と指示した.今回の実験では,このような被験者に対する対応は準備していなかった.

男性のうち多くは,動画の撮影中に笑いを我慢し,撮影が終了したあと吹き出すように・こらえていた笑いを解放するように笑った.実験室という環境の緊張や,羞恥心などの抑制的な感情から,動画の撮影中に笑うことを我慢していたと報告した男性がいた.これに比べて女性は動画の終了を待たず笑顔になった場合が多かった.

\subsection{まとめ}

本システムにおいて,条件1(青年笑い声),条件2(幼児笑い声)で笑顔を誘発できていること,さらに対照とした条件3ではその効果が現れないことを検証した.さらに男女差を分析した.また,男性青年の笑い声よりも幼児の笑い声のほうが笑顔を誘発しやすい傾向が女性では強かった.

また,笑顔の誘発にはディレイが存在し,誘発された笑顔が最も顕著になるのは条件1, 条件2ともに音声再生からおよそ1.5秒後であることがわかった.写真撮影システムを構築する場合は,この結果を元にシャッターを切る時間を設定すれば良い.ただし,この値は再生する音声によって大きく異なるだろうから,異なる音声を扱う場合には今回のような実験を繰り返す必要がある.

比較可能な尺度としてRekognition APIの笑顔尺度を利用した.Rekognition を公開しているOrbeus社は,顔認識エンジン専業の企業で,多数の企業に顔認識システムを提供している.ただし,笑顔度分析の科学的基礎づけは公開されておらず,この値の信頼性に対して評価を加えることは難しい.しかし連続した表情の変化に対して,ほぼ連続した値が得られていることから,ある程度信頼できるものと考えられる.

%\section{実験2}
%
%\subsection{目的}
%
%実験1では,3種類の音声を呈示しながら撮影した写真についてコンピュータビジョンの技術を使って笑顔尺度を測定し,この値を比較することで,我々のシステムが有効に笑顔を誘発できていることが分かった.しかし実験1だけでは,今回構築したカメラシステムが我々の考える「自然な笑顔」であることはわからない.
%
%本実験では,非撮影者とは異なる評価者に撮影された画像を見せ,笑顔の自然さを評価させることで,我々のシステムが自然な笑顔を誘発できているかどうかを検証することを目的とする.
%
%\subsection{実験条件および実験設備環境}
%
%node.js を用いてWebサーバを構築した.評価者は各々のPCのWebブラウザを通じて評価実験に参加した.画面の大きさの条件などは統制できなかったが,ウインドウを最大化してから実験を進めるように指示した.
%
%\begin{figure}[h]
%\begin{minipage}{0.49\columnwidth}
%\begin{center}
%\includegraphics[width=35mm, bb=0 0 980 991]{images/ss/seekbar.png}
%\caption{評価システム (実験A)}
%\end{center}
%\end{minipage}
%\begin{minipage}{0.49\columnwidth}
%\begin{center}
%\includegraphics[width=35mm, bb=0 0 980 991]{images/ss/eval.png}
%\caption{評価システム (実験B)}
%\end{center}
%\end{minipage}
%\end{figure}
%
%\subsection{手続き}
%
%実験A: 被験者は,まず16人,3条件の計48組の動画をさらに5fpsでサンプリングした動画について,それぞれシークバーを前後させながら最も自然な画像のところで止め,「次へ」ボタンを押すことを繰り返す.
%
%実験B: 被験者は16人それぞれについて,3条件のうち2条件を取り出した写真2枚の組 (動画シーケンスのうち実験Aで被験者が選んだ写真を使った) について,どちらが自然な笑顔であるかを判定した.割り当てられた写真は同じ人物の,顔全体・鼻より上・鼻より下の3パターンがあった.被験者は,$16\times{}_3C_2\times3=144$ 組の写真について判定を行ったことになる.
%
%\subsection{結果}
%
%ToDo.
%
%\begin{enumerate}
% \item 実験Aで,最も自然だとされたタイミングのヒストグラムを書く.
% \item 実験Bで,2群検定を行う.
% \item Rekognition で選択を行った場合と,実験Bで人間が選択を行った場合で,違いを調べた.
%\end{enumerate}

\section{結論}

本論文では、シャッターを切る前に笑い声を再生することで自然な笑顔を撮影するカメラシステムを提案し,スマートフォン上で動作するアプリケーションとして実装した.このシステムの有効性を検証するべく,コンピュータビジョンを用いた笑顔尺度測定で,実際に笑顔が撮影できたことを確認した.しかし,その笑顔が「自然である」かどうかは本稿では検証できなかった.

今後は以下について検討する.

\begin{itemize}
\item 今回コンピュータビジョンを用いて評価した撮影された笑顔画像の自然さを,人間の評価者に評価させる実験を計画している.まず撮影された画像群をシークさせながら最も自然な笑顔が撮影された音声再生からの時間遅れを特定し,さらに各条件について自然さを2択強制選択によって評価させる.
\item 今回利用したモーダルは聴覚のみだったが,笑い誘発効果は笑い行動をしている動画でも引き起こせるはずである.インカメラを用いて,笑っている人の動画を見せて笑顔を誘発するカメラは容易に実現できる.音声のみの場合と効果を比較したい.
\item Kleinke らは,自身の笑顔を鏡を通して見ることでポジティブな気分が増幅されることを実験で示した\cite{kleinke1998effects}.Yoshidaらは,変形させた自分の表情をフィードバックすることで,情動体験をポジティブ/ネガティブに操作できることを示した.誘発された笑顔を非撮影者にリアルタイムに呈示することで,ポジティブな情動体験をも誘発することが可能だろう.
\item 本システムでは,被撮影者から見て笑い声の主体がはっきりしなかった.ぬいぐるみに本システムを埋め込んで,ぬいぐるみのキャラクターの笑い声の音声を流すなどして,主体をはっきりと認知させるとどういった効果が起こるのかを調査したい.
\item 本システムはテーマパークや観光地での顔ハメなど,もともと記念撮影行為が多く見られるシチュエーションでの利用に適している.そのような場面で着ぐるみやアトラクション,顔ハメ本体にシステムを埋め込むことで,良い笑顔が撮れる記念撮影スポットができるだろう.
\item 本システムで写真を撮影したあと,笑いを誘発された条件下では「ニヤニヤしてしまった」「あまり良い表情ができなかった」という報告をした被験者がいた.この被験者に撮影できた画像を見せたところ,彼が予想していたよりも好ましい笑顔が撮影できていたという感想が得られた.一歩踏み込んだユーザスタディとして,被験者に「どの条件がいちばん自然な笑顔が撮れていると思うか」を予想させたあと,条件との対応を隠したまま実際の写真を見せ,どれがいちばん自然な表情を評価させることで,自身の表情の主観的な評価と,実際に撮影された写真に対する評価のズレを検討したい.
\end{itemize}



\section{参考文献}
\bibliographystyle{jplain}
\bibliography{ec2014}



%\pagebreak%%!!!
%\vspace*{-\baselineskip}%%!!!

%\appendix
%7
%\section{付録の書き方}

%付録がある場合には,参考文献リストの直後にコマンド \|\appendix| に引き続
%いて書く.付録では,\|\section| コマンドが{\bf A.1},{\bf A.2}などの見出
%しを生成する.

%7.1
%\subsection{見出しの例}

%付録の \|\subsetion| ではこのよう見出しになる.

%8
%\section{研究会論文用コマンド}
%\label{sig}

%各研究会論文誌(トランザクション)には各々に固有のサブタイトル,略称,通
%番がある.最終原稿では,以下のコマンドを \|\documentclass| の{\bf オプショ
%ン}とすることで,これらの情報を与える.

%\begin{itemize}
%\item \|PRO|(プログラミング)
%\item \|TOM|(数理モデル化と応用)
%\item \|TOD|(データベース)
%\item \|ACS|(コンピューティングシステム)
%\item \|CDS|(コンシューマ・デバイス\,\&\,システム)
%\item \|TBIO|(Bioinformatics)\footnote{%
%TBIO, SLDM, CVAは英文論文誌であるので和名はない.}
%\item \|SLDM|(System LSI Design Methodology)\footnotemark[5]
%\item \|CVA|(Computer Vision and Applicaitons)\footnotemark[5]
%\end{itemize}

%また英文論文作成の際には \|english| をオプションに追加すればよい.したがっ
%て,\|\documentclass[PRO]{ipsj}| とすれば「プログラミング」の和文用,
%\|\documentclass[PRO,english]| \|{ipsj}| とすれば英文用となる.

%また研究会には「号」と連動しない「発行月」があるため,学会あるいは編集委
%員会の指示に基づき,発行月を
%
%\begin{itemize}\item[]
%\|\setcounter{|{\bf 月数}\|}{<発行月>}|
%\end{itemize}
%
%によって指定する.

%この他,以下の各節で示すように,いくつかの論文誌に固有の機能を実現するた
%めのコマンドなどが用意されている.

%\newpage%%

%9
%\section{各分冊固有コマンド}

%各分冊によってそれぞれ細かい仕様が違うため,同じコマンドでも出力結果が異
%なる場合がある.また「再受付」,「再々受付」が入る場合があり,それらは

%\noindent
%和文では
%\begin{itemize}\item[]
%\|\|{\bf 再受付}\|{<年>}{<月>}{<日>}|\\
%\|\|{\bf 再再受付}\|{<年>}{<月>}{<日>}|
%\end{itemize}
%英文では
%\begin{itemize}\item[]
%\|\|{\bf rereceived}\|{<年>}{<月>}{<日>}|\\
%\|\|{\bf rerereceived}\|{<年>}{<月>}{<日>}|
%\end{itemize}
%とプリアンブルに追加する.

%9.1
%\subsection{\<「プログラミング(PRO)」固有機能}

%\<「論文誌:プログラミング」には論文以外に,プログラミング研究会での研究
%発表の内容梗概が含まれている.この内容梗概は,\|\documentclass|のオプショ
%ンとして\|abstract|を指定する.\ref{config}~節の\|\maketitle|までの内容
%からなるファイル(すなわち本文がないファイル)から生成する.なお\|\|{\bf 
%受付}や\|\|{\bf 採録}は不要であるが,代わりに発表年月日を,

%\noindent
%和文では
%\begin{itemize}\item[]
%\|\|{\bf 発表}\|{<年>}{<月>}{<日>}|
%\end{itemize}
%英文では
%\begin{itemize}\item[]
%\|\|{\bf Presents}\|{<年>}{<月>}{<日>}|
%\end{itemize}
%により指定する.

%9.1
%\subsection{\<「データベース(TOD)」固有機能}

%\<「論文誌:データベース」の論文の担当編集委員は,
%\begin{itemize}\item[]
%\|\Editor{<氏名>}|
%\end{itemize}
%により指定する.和文では「担当編集委員」,英文では「Editor in Charge:」
%と入る.

%またスタイルの変更に伴い,\underline{本文の最後}に入るので,
%\|\end{document}|の前に直接置く.

%9.2
%\subsection{\<「コンシューマ・デバイス\,\&\,システム(CDS)」固有機能}

%\<「論文誌:コンシューマ・デバイス\,\&\,システム」では,
%論文の種類によって見出しが変わるため,
%オプションで切替えを行う.

%各種別は
%\begin{itemize}
%\item \|systems  |コンシューマ・システム論文\\
%\|         |Paper on Consumer Systems

%\item \|services |コンシューマ・サービス論文\\
%\|         |Paper on Consumer Services

%\item \|devices  |コンシューマ・デバイス論文\\
%\|         |Paper on Consumer Devices

%\item \|research |研究論文\\
%\|         |Research Paper
%\end{itemize}
%となる.

%和文のコンシューマ・システム論文なら,\\
%\|\documentclass[CDS,systems]{ipsj}|
%となり,英文原稿なら \|english|を追加すればよい.

%9.3
%\subsection{\<「Bioinformatics(TBIO)」固有機能}

%Trans.\ Bioinformatics (TBIO)は英文論文誌であるので,\|TBIO|オプションの
%指定によって自動的に\|english|オプションが指定されたものとみなされ,
%\|english| オプションの省略が可能.

%論文種別は以下の3種.
%\begin{itemize}
%\item \makebox[4.9zw][l]{指定なし} Original Paper (Default)
%\item \|Data     | Database/Software Paper
%\item \|Survey   | Survey Paper
%\end{itemize}

%\|\documentclass[TBIO]{ipsj}|でOriginal Paper,\\
%\|\documentclass[TBIO,Survey]{ipsj}|でSurvey Paperとなる.

%また,担当編集委員はTOD同様,\|\Editor|で定義するが,「Communicated by」
%となる.TOD同様,\|\end{document}|の前に直接置く.

%9.4
%\subsection{\<「Computer Vision and Applicaitons\\\<(CVA)」固有機能}

%Trans.\ CVAも英文論文誌であるため,\|english| オプションの省略が可.

%論文種別は3種類あり,
%\begin{itemize}
%\item \makebox[4.9zw][l]{指定なし} Regular Paper (Default)
%\item \|Research | Research Paper
%\item \|system   | Systems Paper
%\end{itemize}
%となる.

%TBIO同様,担当編集委員が入り,
%挿入文章もTBIO同様,「Communicated by」となる.

%9.5
%\subsection{\<「System LSI Design Methodology(SLDM)」固有機能}

%Trans.\ SLDMも英文論文誌であるため,\|english| オプションの省略が可.

%論文種別は2種類あり,
%\begin{itemize}
%\item \makebox[4.9zw][l]{指定なし} Regular Paper (Default)
%\item \|Short    | Short Paper
%\end{itemize}
%となる.

%SLDMも担当編集委員が入るが挿入文章が論文によって自動挿入文章が異なる.

%通常は「Recommended by Associate Editor:」,\|invited|のオプションが入っ
%た場合のみ,「Invited by Editor-in-Chief:」となる.



%% 以下は無視されます

\begin{biography}
\profile{m}{情報 太郎}{1970年生.1992年情報処理大学理学部情報科学科卒.
1994年同大大学院修士課程了.同年情報処理学会入社.オンライン出版の研究
に従事.電子情報通信学会,IEEE,ACM 各会員}
%
\profile{n}{処理 花子}{1960年生.1982年情報処理大学理学部情報科学科卒.
1984年同大大学院修士課程了.1987年同博士課程了.理学博士.1987年情報処
理大学助手.1992年架空大学助教授.1997年同大教授.オンライン出版の研究
に従事.2010年情報処理記念賞受賞.電子情報通信学会,IEEE,IEEE-CS,ACM
各会員}
%
\profile{s}{学会 次郎}{1950年生.1974年架空大学大学院修士課程了.
1987年同博士課程了.工学博士.1977年架空大学助手.1992年情報処理大学助
教授.1987年同大教授.2000年から情報処理学会顧問.オンライン出版の研究
に従事.2010年情報処理記念賞受賞.情報処理学会理事.電子情報通信学会,
IEEE,IEEE-CS,ACM 各会員}
%
\end{biography}



\end{document}
